\documentclass[%
  reprint,
  aps,
  pra,
  superscriptaddress,
  a4paper
]{revtex4-2}

\usepackage{xcolor}
\usepackage{lipsum}


%%%%%%%%%%%%%%%%%%%%%%%%%%%%%%%%%%%%%%%%%%%%%%%%%%%%%%%%%%%%%%%%%%%%%%%%%%%%%%%%%%%%%%%%%%%%%%%%%%%%%%%%%%%%%%%%%%%
%%%%%%%%%%%%%%%%%%%%%%%%%%%%%%%%%%%%%%%%%%%%%%%%%%%%%%%%%%%%%%%%%%%%%%%%%%%%%%%%%%%%%%%%%%%%%%%%%%%%%%%%%%%%%%%%%%%
\usepackage[
  bookmarks=false,
  bookmarks=true,
  colorlinks,
  linkcolor=blue,
  urlcolor=blue,
  citecolor=blue,
  plainpages=false,
  pdfpagelabels,
  final,
  breaklinks=true
]{hyperref}
\hypersetup{
pdftitle={The Arthur BibTeX style file}, 
pdfauthor={Emilio Pisanty}
}
%%%%%%%%%%%%%%%%%%%%%%%%%%%%%%%%%%%%%%%%%%%%%%%%%%%%%%%%%%%%%%%%%%%%%%%%%%%%%%%%%%%%%%%%%%%%%%%%%%%%%%%%%%%%%%%%%%%


\usepackage{natbib}
%% reduce spacing inside [1, 2]:
\makeatletter \def\NAT@def@citea{\def\@citea{\NAT@separator\,}} \makeatother
\newcommand{\citer}[1]{Ref.~\citealp{#1}}
\newcommand{\citers}[1]{Refs.~\citealp{#1}}


\newcommand{\reffig}[1]{Fig.~\ref{#1}}



\begin{document}



\title{The Arthur style for BibTeX}

\author{Emilio Pisanty}
\email{emilio.pisanty@kcl.ac.uk}
\affiliation{Attosecond Quantum Physics Laboratory, Department of Physics, King's College London, Strand, WC2R 2LS London, UK}

\date{\today}

\begin{abstract}
This is a showcase file for the BibTeX style file \texttt{arthur.bst}.
\end{abstract}

\maketitle

The BibTeX style file \texttt{arthur.bst} is optimized for physics (more specifically atomic, molecular and optical physics). It was generated from the \texttt{makebst} utility, and then modified by hand to give it DOI linking capabilities comparable to those used in production by modern physics journal

It also admits arbitrary eprint capability for indicating the URL and identifier of the eprint,
\begin{verbatim}
  eprint = {https://arxiv.org/abs/yymm.nnnn},
  archive = {arXiv:yymm.nnnn},
\end{verbatim}
whether that be on arXiv or somewhere else, and it has a dedicated \texttt{preprint} entry type:
\begin{verbatim}
@preprint{arxivPreprint,
  author = {Ann Author},
  title = {A preprint still under review},
  year = {2020},
  eprint = {https://arxiv.org/abs/yymm.nnnn},
  archive = {arXiv:yymm.nnnn},
}  
\end{verbatim}

This file exists to showcase how the style file behaves when typesetting 
journal articles~\cite{journalArticle},
preprints~\cite{arxivPreprint}, 
books~\cite{bookExample}, 
theses~\cite{thesisExample}, 
code~\cite{codeExample}, and
collections~\cite{collectionExample}.
This repository also contains two shorter versions of the style file, \texttt{arthurShort.bst} and \texttt{arthurUltraShort.bst}, which produce minified entries.


The name `arthur' derives from the name of the `master' style form the \texttt{makebst} utility, \texttt{merlin.bst}, in the understanding that \texttt{arthur} is still learning from \texttt{merlin}.

On the whole, I am extremely happy with this style file, though it does have a minor `tick' in that it tends to hyperlink the period between a paper's title and its journal reference; this is a side effect of the DOI linker which I was unable to iron out. 
If this is found undesirable, it can be fixed using a simple Perl script; a sample `prettifier' shell file is provided.

For some historical context, I prepared this style file around 2015, before version 4.2 of the \texttt{revtex} package became widespread. 
The updated version of \texttt{revtex} does provide APS style files which include proper DOI hyperlinking, with the clearest contender being \texttt{apsrev4-2.bst}, and those may be found preferable depending on individual tastes; copies are easily found online, and they are natively available to \texttt{revtex4-2} documents. 
Nevertheless, a copy of \texttt{apsrev4-2.bst} is included in this repository (under its LPPL license) for ease of comparison.


%\bibliographystyle{apsrev4-2}
\bibliographystyle{arthur}
\bibliography{references}{}



\end{document}













